\documentclass[]{article}
\parindent 0.0cm
\parskip 0.2cm
\textwidth 6.2in
\textheight 8.1in
\oddsidemargin 0pt
\evensidemargin \oddsidemargin
\marginparwidth 0.5in

\begin{document}

\begin{center}
\LARGE \bf
Communication Buffer Management in the Alliance Cache Kernel Environment
\normalsize
\end{center}

{ \it 
{\bf Abstract: } This document describes the implementation of
communication in the Alliance CK environment, and details the buffer
management component thereof as implemented in the Kernel Libraries. }


Ramon van Handel {\tt <vhandel@chem.vu.nl>},\\
April 16, 1999\\
Revised May 8, 1999

\section{Introduction}
The supervisor component of the Alliance Core System is the Alliance Cache
Kernel. The CK provides a very minimal amount of services, all of which
are low-level:  scheduling, physical memory allocation, mapping of virtual
memory, and signaling.  All other services are performed at the level of
external kernels, including virtual memory management and communication.

Unlike other microkernel-like systems, the CK does not provide message
passing primitives.  In stead, the CK allows for shared physical memory as
the sole means of communication between address spaces with different
owning external kernels in the core system.  At a higher level, the core
system provides external kernels with a fast CORBA Object Request Broker
which is optimised for local communications.

Shared memory, though, is not directly suitable for usage in a system
based around message-passing abstractions, such as CORBA.  It allows two
threads in different address spaces pass information to each other without
kernel intervention, but the format of this information is left undefined
by the CK. In order to facilitate passing messages over the shared memory
interface, the core system provides a message buffer management layer as
part of its kernel libraries.  The usage of this layer is detailed in the
rest of this document.

\section{Basics of the Message Buffering Layer}
The principle (and the implementation, as well) of the buffering layer is
very simple. The buffering layer is built on top of the memory allocation
library provided by the kernel libraries (free list allocation.) The free
list is kept {\it inside\/} the shared memory area, and is protected by a
shared mutex. The shared memory area is represented by the following
structure:

\begin{verbatim}
#define SIGCHANNEL 0x4143686e
typedef struct {
    UDATA signature;
    DESCRIPTOR shmem;
    DATA attached;
    Mutex bufMutex;
    KLfreeListDescriptor bufFree;
    UBYTE data[0];
} *MsgChannel;
\end{verbatim}

{\tt UDATA signature }is used for validity checking. {\tt DESCRIPTOR shmem
}is the descriptor of the shared memory object.  It is used by the
buffering layer to communicate with the CK. {\tt DATA attached} are the
amount of threads that are attached to the channel. {\tt Mutex bufMutex
}is the mutex that protects the freelist from being corrupted. {\tt
bufFree }is the freelist, as used by the freelist allocation library. The
part of the area that is free for messages starts at {\tt data}.

Initially, all of the shared memory area but the header (that is, from
data to the end of the initially allocated page) is freed in the message
channel's freelist.  Once the connection is set up with (an) other
thread(s), each thread can allocate a message buffer inside the message
channel. A message buffer looks like this:

\begin{verbatim}
#define SIGBUFFER  0x41427566
typedef struct {
    UDATA signature;
    UADDR size;
    UBYTE data[0];
} *MsgBuf;
\end{verbatim}

{\tt UDATA signature }is used for validity checking.  {\tt UADDR size }is
the size of the message buffer. The message contents are from {\tt data
}to {\tt (data+size)}.  If the message channel isn't big enough, the
allocation function will attempt to grow the associated shared memory area
until there's enough room for the allocated buffer.  Message areas can be
freed again when they're not being used anymore.

Typically, there will be two message buffers in use by a pair of
synchronously communicating applications, one for each direction, but this
is completely in control of the applications in question.  Certainly with
asynchronous communication many more buffers could be in use. Whether
allocated buffers are reused or not is also up to the applications - it
depends entirely on the communication protocol used by the applications.

Aside from buffer allocation functions, the message buffer layer also
provides functions to signal applications when a message has been written
to a buffer synchronously (blocking for reply) or asynchronously.

\section{The Message Buffering Layer API}

\subsection{Initialisation}

\subsubsection{Initialising the Message Buffering library}

XXX --- TODO

\subsubsection{Opening a new Message Channel}

In order to open a new message channel, the message buffering library
provides the function:

{\parindent0.6cm\tt MsgChannel KLopenMsgChannel(UADDR inisize);}

This function registers a new shared memory area with the CK with the
initial size of one page if {\tt inisize} is 0 or of {\tt inisize}
bytes round up to page boundary, and sets up the message channel data
structure inside it.  It also frees up the rest of the shared memory area
for buffer allocation.

\subsubsection{Closing a Message Channel}

If you are the owner\footnote{The owner of a shared memory area or
message channel is the thread that created it.} of a message channel you
can close the channel by using the function

{\parindent0.6cm\tt VOID KLcloseMsgChannel(MsgChannel chan);}

where {\tt chan} is the message channel to close.  This will free up the
physical memory used by the channel and it will also notify all connected
threads that the connection was terminated.

\subsection{Connecting and Disconnecting}

\subsubsection{Connecting to a thread}

Before you can communicate with a thread, you will need to `connect' an
open message channel to it.  This is done using the function

{\parindent0.6cm\tt BOOLEAN KLmsgConnect(MsgChannel chan, DESCRIPTOR thread);}

where {\tt chan} is the channel to connect to and {\tt thread} is the
thread you want to communicate with.  The {\tt KLmsgConnect()} function
will block until the thread has been connected (return value = {\tt TRUE})
or the connection has been rejected (return value = {\tt FALSE}).

When a thread issues the connect request, the thread it wants to connect
to is signalled by the CK (invite signal).  If the thread wants to accept
the connection, it calls

{\parindent0.6cm\tt MsgChannel KLmsgAccept(DESCRIPTOR shmem);}

{\tt shmem} is the descriptor of the shared memory area, which the thread
gets as the signal parameter.  This will sucessfully set up the connection
and return the location of the message channel in the (accepting) thread's
address space.  In order to let the inviting thread know that it
connected, it increases the {\tt attached} variable in the channel
structure and sends an asynchronous message to the inviting thread which
contains the address of the whole channel (in stead of a message buffer
allocated inside the channel.)  The inviting thread will recognise the
channel signature, note that the attached threads count has increased, and
thus know that the thread has attached.

If, on the other hand, the thread wishes to reject the connection, it
calls:

{\parindent0.6cm\tt VOID KLmsgReject(DESCRIPTOR shmem);}

{\sl [XXX --- think of a good reject protocol; a rejecting thread doesn't
have the channel address.  Also, without connecting it cannot signal to
the inviting thread.  This needs thought.]}

If a thread tries to connect to a ``misbehaving'' thread, which doesn't
issue a {\tt KLmsgReject()} call or an {\tt KLmsgAccept()} call, the
thread will hang.  In order to avoid this the thread can also use the
function

{\parindent0.6cm\tt BOOLEAN KLmsgConnectTimeout(MsgChannel chan, DESCRIPTOR thread, TIME timeout);}

which is basically the same as {\tt KLmsgConnect()}, except for the fact
that this function will also return {\tt FALSE} if a time {\tt timeout}
has passed without a connection having been established or rejected.

\subsubsection{Disconnecting from a message channel}

At any time, a connected thread can disconnect itself from a message
channel using the call

{\parindent0.6cm\tt VOID KLmsgDisconnect(MsgChannel chan);}

In order to let the other connected threads know that it has disconnected,
it first decreases the attached threads count in the channel structure and
then sends an asynchronous message to the channel which contains the
address of the whole channel, like when attaching.  On recognising the
channel signature and the decreased attached count, the still connected
threads will know that a thread has disconnected.  After this, the thread
tells the CK to detach it from the shared memory area.

If the owner of a channel tries to disconnect from the channel, it is
equivalent to closing the channel. One thread cannot kick another
connected thread off a channel, except if the whole channel is closed by
the owner.

\subsection{Buffer Allocation}

In order to allocate a message buffer in a message channel, the following
call is provided:

{\parindent0.6cm\tt MsgBuf KLallocMsgBuf(MsgChannel chan, UADDR size);}

where {\tt chan} is the channel to allocate the buffer in and {\tt size}
is the size of the requested message buffer.  If the requested size
exceeds the amount of free memory in the message channel,
{\tt KLallocMsgBuf()} will attempt to grow the message channel.  If it
doesn't succeed, it returns {\tt NIL}.

In order to free an allocated message buffer, call

{\parindent0.6cm\tt BOOLEAN KLfreeMsgBuf(MsgChannel chan, MsgBuf buffer);}

This will not attempt to shrink the shared memory area if there's an
unused physical page - call the garbage collecting function for that.

{\tt KLfreeMsgBuf()} should return {\tt TRUE}.  If it doesn't, this means
that the message buffer was somehow corrupted.  Should this function
return {\tt FALSE}, it is wise to immediately close the message channel
lest there be more memory corruption somewhere.

\subsection{Signalling functions}

Once a message has been written to an allocated message buffer, it is
possible to signal the thread connected to the message channel that the
message has been written. {\it This is not strictly neccessary;\/} the
threads might have a different way of notification, such as polling, some
external event, or a timeout.  Unlike the rest of the messaging procedure
signalling requires a trap to the kernel, so it is possible to skip this
step if the situation provides for a way that consumes less time.

The message buffering library provides three signalling calls:

{\parindent0.6cm\tt VOID KLmsgSendSync(MsgChannel chan, MsgBuf buffer);

\parskip0.04cm BOOLEAN KLmsgSendSyncTimeout(MsgChannel chan, MsgBuf buffer, TIME timeout);

VOID KLmsgSendAsync(MsgChannel chan, MsgBuf buffer);}

The first two functions are synchronous calls:  they block the thread
until a reply has been received from the message channel that was
signalled.  In the case of {\tt KLmsgSendSyncTimeout()} the communication 
might also time out after the specified amount of time, in which case the
function returns {\tt FALSE}.  The asynchronous call returns directly and
does not block, thus providing for asynchronous communication (the thread
will receive a signal when the reply has arrived.)

\subsection{Garbage collection}

XXX --- TODO

\section{Evaluation}

The Message Bufferring layer has not yet been evaluated.

\section{Conclusion}

The Message Buffering library, which is part of the Alliance Kernel
Libraries, provides a buffering layer in order to ease the use of
message-oriented protocols, such as those used in the System ORB, on top
of shared memory, which is the sole means of communications within the
Alliance Cache Kernel environment.  It is my hope that its use will
simplify the design and implementation of system components that require
these services.

\end{document}

